\newif\ifdebug
%\debugtrue
\ifdebug
\documentclass[12pt,a4paper]{ctexrep}
\usepackage[a4paper, portrait, margin=0.8in]{geometry}
\usepackage{graphicx} % Required for inserting images
\usepackage{amsmath}
\usepackage{amsfonts}
\usepackage{amssymb}
\usepackage{amsthm}
\usepackage{markdown}
%\usepackage{china2e} 
\usepackage[utf8]{inputenc}
\begin{document}
\fi

\chapter{Game Theory游戏论}
Nim and the Sprague-Grundy Theorem, One-shot Games and Nash Equilibria, Two-player Infinite-duration Games on Graphs
\section{定义}
\subsection{基本定义}
Topological Ordering: Given a DAG $G = (V,E)$, a topological ordering is a permutation $\pi$ of vertices such that $\forall e = (u,v)$, the vertex $u$ appears before $v$ in $\pi$.

Dominant Strategy: A strategy $s_i \in S_i$ is dominant if $\forall s_i, u_i(s_i,s_{-i}) \geqslant u_i(s_{i'},s_{-i'})$. A rational player is always going to take the dominant strategy.

Equilibrium State: A state where no one wants to change. (i.e. If only one person changes, then his payoff will decrease)

Nash Equilibrium: Nash Equilibrium is a concept that occurs when each player in a non-cooperative game chooses and stays with their optimal strategy in response to knowing other players' anticipated strategies. Also, no player in a Nash equilibrium has a dominant strategy.

Pure Nash Equilibrium: An outcome $s = (s_1,s_2,\dots)$ is a nash equilibrium iff $\forall i, \forall s'_i, u_i(s_i,s_{-i}) \geqslant u_i(s'_i,s'_{-i})$. Nobody plays randomly in a Pure Nash Equilibrium.

Mixed Nash Equilibrium: Somebody plays randomly, the expectation of other players cannot be increased by playing another strategy.

$\divideontimes$ Nash Equilibrium only applies to games with finite players.\\

Arena: A directed finite graph $G = (V,E,V_1,V_2)$ such that $\forall v \in V$, outdegree($v$)$\geqslant 1$, and $V_1 \cup V_2 = V$, $V_1 \cap V_2 = \emptyset$.

Game: a arena $G = (V,E,V_1,V_2)$ and a starting vertex $v_0\in V$

Strategy: A strategy for player $i$ is a function $\sigma_i = V^* \times V_i \mapsto V$. If the strategy is memoryless, than $V^*$ is  nothing.

Outcome: An outcome is an infinite walk on $G$ starting at $v_0$. $o(\sigma_1)$ is the outcome for the player who uses $\sigma_1$ as his strategy. $O$ is all outcomes of the game.

objective: An objective for player 1 is a set $Obj_1 \subseteq O$.

Zero-Sum game: $Obj_2 = O \setminus Obj_1$

Winning Strategy: A strategy $\sigma_1$ for player 1 is a winning strategy if for $\forall \sigma_2, o(\sigma_1,\sigma_2) \in Obj_1$. The set of initial states from which player i has a winning strategy is $Win_i$.

Determined: A game is determined if for $\forall v_0$ either player 1 or player 2 has a winning strategy.
\section{重要游戏}
\subsection{Nim}
Two numbers, lose if one can't decrease any heap.

Turn-based, finite game, impartial.

We can model these kind of games with directed graphs, where each state in the game corresponds a vertex and a state change corresponds a edge. Assign a "W" or "L" to every vertex. A vertex $v$ is "W" if there's an outgoing edge to "L", but $v$ is "L" if all outgoing edges lead to "W".

Bitwise xor($\oplus$): In base 2, output "1" if the two digits are not the same and output "0" if the two digits are the same. Apply this to every digit of two numbers.

General Nim: We have number $a_1,a_2,\dots,a_n \in \mathbb{N}$ and each player can choose a number and decrease it in their turn. Then $L = \{(a_1,a_2,\dots,a_n)|\oplus_{i=1}^n a_i = 0\}$ and $W = \{(a_1,a_2,\dots,a_n)|\oplus_{i=1}^n a_i \neq 0\}$

\subsection{One-Shot Games}
A one-shot game consists of a set $S_i$ of strategies for each player, a payoff function $u_i: S_i \mapsto \mathbb{R}$. Each player chooses a strategy and the outcome of the game is set. Assume every player is rational, i.e. they choose to maximize their own payoff.
\subsubsection{Prisoner's Dilemma}
\begin{center}
\begin{tabular}{|c|c|c|}
\hline
A  B & Confess & Silent\\
\hline
Confess & 4 4 & 1 5 \\
\hline
Silent & 5 1 & 2 2 \\
\hline

\end{tabular}
Both want to minimize their prison time.
\end{center}
Then both will choose to confess because no matter what the other one choose, confessing gives them the least prison time.

\subsubsection{Pollution Game}
If a country pollutes, then he doesn't cost anything additional, but every country will cost "1". If a country does not pollute, then he will cost "5", and every country else will not cost anything. If everyone is rational, then everyone will pollute.

\section{定理/结论}
Every DAG have a topological ordering.

\subsection{Sprague-Grundy Theorem}
Assign a unber to every vertex in $G$: If $v$ has no outgoing edges, then assign $nim(v) = 0$. If $v$ has outgoing edges to $u_1,u_2,\dots,u_n$, then $nim(v) = max\{nim(u_1),nim(u_2),\dots, nim(u_n)\}+1$. Then $L = \{(a_1,a_2,\dots,a_n)|\oplus_{i=1}^n nim(a_i) = 0\}$ and $W = \{(a_1,a_2,\dots,a_n)|\oplus_{i=1}^n nim(a_i) \neq 0\}$

\section{Algorithms}
\subsubsection{Reachability game}
Player 1's objective is to reach $T \subseteq V$

Rule 1: $T \subseteq Win_1$

Rule 2: If $v$ is a player 1 vertex, then $\exists (v,u) \in E, u \in Win_1 \Rightarrow v \in Win_1$

Rule 3: If $v$ is a player 2 vertex, then $\forall (v,u) \in E, u \in Win_1 \Rightarrow v \in Win_1$

This construction algorithm will terminate because there's finite number of vertices, $|Win_1|$ can't increase infinitely.

Process: $T_0 = T$. $T_i = T \cup \{v \in V_1|\exists (v,u)\in E, u \in T_{i-1}\} \cup \{v \in V_2| \forall (v,u)\in E, u \in T_{i-1}\}$

$Win_1$ = $\cup T_i$, denote it as $T_k$

$Win_2$ = $V \setminus Win_1$

Attractor: $Attr_1 = \cup T_i$
\subsection{B\"uchi game}
Player 1's objective is to reach $T \subseteq V$ infinitely many times. 

By "Reachability Game", we are able to find $A_1$ = $Attr_1(T)$(Attracting player 1), $B_1 = A_1^C$(Trapping Player 1), and $C_1 = Attr_2(B_1)$ be the set of vertices that Player 2 can force the game to visit $B_1$, thus trapping Player 1, not letting Player 1 visit $T$ infinitely many times.

Inside $C_1$, Player 2 will not let Player 1 achieve it's goal. Outside $C_1$, for player 1 vertices, player 1 has a outgoing edge that will not enter $C_1$ and will choose it; player 2 does not have a outgoing edge leading to $C_1$ (or it'll be in $C_1$).

Thus, removing $C_1$ part of the whole graph will not affect the final answer to whether player 1 can achieve its goal or not. Remove $C_1$ and we have a smaller game $G_1 = G - C_1$.

Repeat the process using $G_1$ and the same $T$. This process will terminate as $|G|$ is getting smaller every iteration and that $|G|$ is finite.

At the end, $Win_2 = \cup_i C_i$, $Win_1 = V \setminus Win_2$\\

%\begin{figure}
\centering
\includegraphics[scale=0.5]{buchi_game.png}
%\end{figure}
\section{重要定理详解}
%可补充
\ifdebug
\end{document}
\fi
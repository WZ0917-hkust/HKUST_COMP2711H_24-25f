\newif\ifdebug
%\debugtrue
\ifdebug
\documentclass[12pt,a4paper]{ctexrep}
\usepackage[a4paper, portrait, margin=0.8in]{geometry}
\usepackage{graphicx} % Required for inserting images
\usepackage{amsmath}
\usepackage{amsfonts}
\usepackage{amssymb}
\usepackage{amsthm}
\usepackage{markdown}
%\usepackage{china2e} 
\usepackage[utf8]{inputenc}
\begin{document}
\fi
\chapter{Propositional logic逻辑学、数论基础}
Proposotional Logic, Predicate Logic and Peano's Axioms, Induction, Well-ordering and Infinite Descent, Induction Problems and Strong Induction.
\section{定义}
\subsection{Operations}
\textbf{Boolean Algebra:}

\begin{tabular}{l l}
Negation: &"$\neg$"\\
Disjunction: &"$\vee$" or logical OR\\
Conjunction: &"$\wedge$" or logical AND\\
Implication: &"$\rightarrow$"\\
XOR: &"$\oplus$" $\rightarrow$ $a\oplus b$ means if $a == b$\\

\end{tabular}

$p \vee (q \wedge r) = (p \vee q) \wedge (p \vee r)$

$p \wedge (q \vee r) = (p \wedge q) \vee (p \wedge r)$

\textbf{Quantifiers:}

Universal quantifier: $\forall$

$\indent$Existential quantifier: $\exists$\\

$\neg \forall x, p(x) \iff \exists x, \neg p(x)$

$\neg \exists x, p(x) \iff \forall x, \neg p(x)$\\

$\forall n,m \in N, n \leq m \iff \exists x \in N, n+x = m$

\subsection{Definitions of 专有名词}
Tantology:恒成立

Predicate:猜想

Proposition:命题

Contradiction:矛盾

\section{公理}
Peano's Axioms:自然数公理 $\rightarrow$ 定义自然数
\section{定理/结论}
$\sqrt 2$ is irrarional. $\leftarrow$ proof by assumpution.

There is infinite amount of prime numbers. $\leftarrow$ proof by assuming finite amount of prime, then $k = \prod_{i=1}^{n}\, p_{i} + 1$, $k = a \times b, a \in \{p_{n}\}$, no infinite descending {$b_{n}$}.\\

\noindent In Fibonnacci numbers, $F_{0} = 0; F_{1} = 1$

$\sum_{i = 1}^{n} F_{i} = F_{n+2}-1$

$\forall n \in N, F_{1}+F_{3}+F_{5}+\dots +F_{2n-1} = F_{2n}-1$

$\forall n \in N, F_{2}+F_{4}+F_{6}+\dots +F_{2n} = F_{2n+1}-1$

$F_{n}$ and $F_{n+1}$ are relatively prime.\\

\noindent $\forall n \geq 1, \, 1^{3}+2^{3}+\dots + n^{3} = (1+2+ \dots +n)^{2}$
\section{重要定理详解}
\noindent Peano's axioms: 

a) 0 is a natural number. 

b) Every natural number has a successor $s(n)$. 

c) For all $n,m \in N$, if $s(n) = s(m)$, then $n = m$. 

d) For every natural number $n$, the successor of n is not 0. 

e) If $K$ is a set such that $0 \in K, \forall n \in N, n \in K \Rightarrow s(n) \in K$, then $K = N$

\noindent "+":

a) $\forall n \in N, n+0 = n$

b) $\forall n,m \in N, n+s(m) = s(n+m)$

\noindent "$\times$":

a) $\forall n, n\times 0 = 0$

b) $\forall n,m \in N, n \times s(m) = n \times m + n$
\section{方法}
\noindent Mathematical induction (数学归纳法): If you want to show $\varphi(n)$ holds for $n \in N$ 

a) Base case: the induction basis holds.($\varphi(0)$ holds)

b) Induction step: when $\varphi(n)$ holds, then $\varphi(n+1)$ holds\\
Strong Induction: proof $\forall n \in N$ , $p(0) \wedge p(1) \wedge \dots \wedge p(n)$ holds $\Rightarrow p(n+1)$.\\
Well-ordering principle: every non-empty subset $A$ of $N$ has a smallest element.\\
Infinite-descent principle: There is no infinite sequence $\{a_{1}, a_{2}, \dots \}$ of natural numbers, such that $\forall i, a_{i} > a_{i+1}$\\
Pigeon-Hole Principle: If we have n holes and n+1 pigeons are put in them, then there exists a hole with at least 2 pigeons.
\ifdebug
\end{document}
\fi